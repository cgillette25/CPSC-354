\documentclass{article}
\usepackage[utf8]{inputenc}   % Allows the use of UTF-8 characters
\usepackage[T1]{fontenc}      % Ensures proper font encoding
\usepackage{amsmath}
\usepackage{amssymb}          % For additional math symbols
\usepackage{mathtools}        % For extended math features
\usepackage{amsthm}
\usepackage{graphicx}
\usepackage{hyperref}         % For clickable links in the Table of Contents
\usepackage{listings}
\usepackage{xcolor}

% Lean Style Definition for Listings
\lstdefinestyle{leanstyle}{
    language=Python,
    basicstyle=\ttfamily\small,
    keywordstyle=\bfseries\color{blue},
    commentstyle=\itshape\color{gray},
    stringstyle=\color{orange},
    mathescape=true,
    escapeinside={(*@}{@*)},
    literate=
    {⟨}{{$\langle$}}1
    {⟩}{{$\rangle$}}1
    {∧}{{$\land$}}1
    {∨}{{$\lor$}}1
    {¬}{{$\lnot$}}1
    {λ}{{$\lambda$}}1
    {→}{{$\rightarrow$}}1
    {∀}{{$\forall$}}1
    {∃}{{$\exists$}}1
    {≠}{{$\neq$}}1
    {≤}{{$\leq$}}1
    {≥}{{$\geq$}}1
    {≡}{{$\equiv$}}1
    {…}{{$\ldots$}}1
    {×}{{$\times$}}1
    {𝔽}{{$\mathbb{F}$}}1
    {ℕ}{{$\mathbb{N}$}}1,
    breaklines=true,
    postbreak=\mbox{\textcolor{red}{$\hookrightarrow$}\space},
    keepspaces=true,
    columns=flexible,
}


\title{CPSC-354 Report}
\author{Chaz Gillette}
\date{\today}

\begin{document}

\maketitle
\tableofcontents
\newpage

\section{Week 1: Introduction to Lean and Natural Number Game Tutorial}
\label{sec:week1}

In week one, we worked with Lean and reviewed our discrete math knowledge in the Number Game Tutorial. Problems and solutions are listed below.

\subsection*{Homework Solutions: Week 1}

\subsubsection*{Level 5}
\begin{lstlisting}[style=leanstyle]
a + (b + 0) + (c + 0) = a + b + c.

rw [add_zero]

rw [add_zero]

rfl
\end{lstlisting}

\subsubsection*{Level 6}
\begin{lstlisting}[style=leanstyle]
a + (b + 0) + (c + 0) = a + b + c.

rw [add_zero c]

rw [add_zero b]

rfl
\end{lstlisting}

\subsubsection*{Level 7: succ\_eq\_add\_one Theorem}
\begin{lstlisting}[style=leanstyle]
Theorem succ_eq_add_one: For all natural numbers a, we have succ(a) = a + 1

rw [one_eq_succ_zero]

rw [add_succ]

rw [add_zero]

rfl
\end{lstlisting}

\subsubsection*{Level 8: 2 + 2 = 4}
\begin{lstlisting}[style=leanstyle]
2 + 2 = 4.

nth_rewrite 2 [two_eq_succ_one]  -- only change the second `2` to `succ 1`.

rw [add_succ]

rw [one_eq_succ_zero]

rw [add_succ, add_zero]  -- two rewrites at once

rw [three_eq_succ_two]  -- change `succ 2` to `3`

rw [four_eq_succ_three]

rfl
\end{lstlisting}

\subsection*{Detailed Explanation: Level 7 Proof}
I chose to explain the proof for level seven because this is where we make the breakthrough with addition. Our goal is to prove that the successor of \(a\) is equal to \(a + 1\). So, in step one, we want to rewrite one as the successor of zero. That gives us \(\text{succ } a = a + \text{succ } 0\). The next step is \texttt{add\_succ} so that \(\text{succ } a = \text{succ } (a + 0)\). After that, we can remove the zero by \texttt{rw [add\_zero]}, which will leave us with \(\text{succ } a = \text{succ } a\), which is thus proven true with the reflexive property.

\section*{Lessons from the Assignments}
\subsection*{Lesson from Week 1}
Week one was our review and introduction to the math side of what we'll be learning this semester. We started by revisiting the basic rules of discrete math. This meant getting back into the flow of writing out our proofs using the rules that we have access to. With only natural numbers to start, we started looking at successors again and eventually proving our way toward addition.

For me, this was a needed refresher because it's been a moment since I took discrete math, and I'm unfamiliar with writing my proofs as code, which is a learning curve for me. I'm a very pen-to-paper mathematician, so thinking about math at the same time I'm trying to recall syntax for code is a challenge for me. That said, we went through eight levels of proofs, and I was able to begin to get the hang of it.

I'm looking forward to bridging the gap between my math knowledge and how I involve it when I code. Sometimes I feel like I have the education to understand the concepts, but I struggle to apply them when I'm coding. The speed in which I type out code is not as quick as how I think about what I'd like to apply. This first assignment was a nice intro to opening my eyes as to what it might look like to get faster at that and also write technical reports in a coding environment as well. By shifting everything I do—the code, the math, the reporting—into an IDE, I know that I'll be able to get more comfortable working in that environment.

\section{Week 2: Finishing the NNG Addition World}
\label{sec:week2}

In week two, we focused on completing the Natural Number Game (NNG) Addition World, which helped solidify our understanding of addition in Lean. The problems and solutions for Levels 1-5 are listed below.

\subsection*{Homework Solutions: Week 2}

\subsubsection*{Level 1: zero\_add}
\begin{lstlisting}[style=leanstyle]
theorem zero_add (n : nat) : 0 + n = n := by
induction n with d hd
rw [add_zero]
rfl
rw [add_succ]
rw [hd]
rfl
\end{lstlisting}

\subsubsection*{Level 2: succ\_add}
\begin{lstlisting}[style=leanstyle]
theorem succ_add (a b : nat) : succ a + b = succ (a + b) := by
induction b with d hd
rw [add_zero]
rw [add_zero]
rfl
rw [add_succ]
rw [hd]
rw [add_succ]
rfl
\end{lstlisting}

\subsubsection*{Level 3: add\_comm}
\begin{lstlisting}[style=leanstyle]
theorem add_comm (a b : nat) : a + b = b + a := by
induction b with d hd
rw [add_zero]
rw [zero_add]
rfl
rw [add_succ]
rw [hd]
rw [succ_add]
rfl
\end{lstlisting}

\subsubsection*{Level 4: add\_assoc}
\begin{lstlisting}[style=leanstyle]
theorem add_assoc (a b c : nat) : a + b + c = a + (b + c) := by
induction c with d hd
rw [add_zero]
rw [add_zero]
rfl
rw [add_succ]
rw [hd]
rw [add_succ]
rfl
\end{lstlisting}

\subsubsection*{Level 5: add\_right\_comm}
\begin{lstlisting}[style=leanstyle]
theorem add_right_comm (a b c : nat) : a + b + c = a + c + b := by
induction c with d hd
rw [add_zero]
rw [add_zero]
rfl
rw [add_succ]
rw [hd]
rw [add_succ]
rw [add_comm b d]
rfl
\end{lstlisting}

\subsection*{Mathematical Proof for Level 5: add\_right\_comm}
The goal is to prove the right commutativity of addition, meaning for all natural numbers \(a\), \(b\), and \(c\), the equation \(a + b + c = a + c + b\) holds. This is done by using induction on \(c\).

\textbf{Base Case:} When \(c = 0\), we need to prove:
\[
a + b + 0 = a + 0 + b
\]
Using the identity property of addition, we know that \(a + 0 = a\) and \(a + b + 0 = a + b\). Thus, both sides simplify to:
\[
a + b = a + b
\]
which is true.

\textbf{Inductive Step:} Assume that \(a + b + c = a + c + b\) holds for some \(c\). Now, we must show that it holds for \(c + 1\):
\[
a + b + (c + 1) = a + (c + 1) + b
\]
Using the associative and commutative properties, and the inductive hypothesis, we can manipulate the expression to prove the equality.

\subsection*{Detailed Explanation: Level 5 Proof (add\_right\_comm)}
In this proof, the goal is to show that for all natural numbers \(a\), \(b\), and \(c\), the equation \(a + b + c = a + c + b\) holds. This is the right commutativity of addition. The proof is done by induction on \(c\).

1. **Base Case (\(c = 0\)):** We need to prove \(a + b + 0 = a + 0 + b\). Since adding zero doesn't change the value, both sides simplify to \(a + b\), which are equal.

2. **Inductive Step:** Assume the statement holds for some \(c\). That is, \(a + b + c = a + c + b\). We need to show \(a + b + (c + 1) = a + (c + 1) + b\).

   - Starting with the left side:
     \[
     a + b + (c + 1) = (a + b + c) + 1
     \]
   - By the inductive hypothesis:
     \[
     (a + c + b) + 1 = a + c + (b + 1)
     \]
   - Since addition is associative:
     \[
     a + (c + (b + 1)) = a + ((c + 1) + b)
     \]
   - Therefore, the right-hand side matches, proving the statement.

\section*{Lessons from the Assignments}
\subsection*{Lesson from Week 2}
In week two, we delved deeper into the relationship between mathematical proofs and Lean code proofs. We learned how to perform proofs recursively, using induction. This requires establishing a base case and then proving that if the statement holds for \(n\), it also holds for \(n + 1\). By doing so, we can prove statements for all natural numbers.

I learned how to translate mathematical induction into Lean code, which strengthened my understanding of both programming and mathematical proof techniques. This experience highlighted the importance of rigorous logical reasoning in coding proofs.

\section{Week 3: Using LLMs for Literature Review}
\label{sec:week3}

In Week 3, I explored the topic of \textbf{Quantum Programming Languages} using an LLM to guide my investigation. I also began to code a computer in Python without the help of any libraries.

\subsection*{Link to the Full Literature Review}
The full literature review, including the questions and answers from the LLM, can be found \href{https://github.com/cgillette/CPSC-354/blob/main/week3/README.md}{here}.

\subsection*{Discord Post}

My Discord name is Chaz Gillette, and below is a copy of my Discord post summarizing the literature review:

\begin{quote}
\small
"What are some of the key differences between classical logic and constructive logic that we should be mindful of when working through the Lean tutorials?"
\end{quote}

\subsection*{Reviews I Voted For}
I voted for the following two reviews after reading them:

\begin{enumerate}
    \item \href{https://github.com/mdrivas/CPSC353-Assignment3/blob/main/README.md}{Review 1}
    \item \href{https://github.com/ATacoDev/LitReview354}{Review 2}
\end{enumerate}

\subsection*{Lessons Learned}
In week 3, I learned a lot about what makes programming for quantum computing different; mostly, this has to do with the non-binary nature of quantum computing compared to current computers. Outside of my report, I learned how to break down order of operations when coding. Using functions to call other functions within parentheses, I began to tackle mathematical equations like a real calculator would.

\section{Week 4: Introduction to Parsing and Context-Free Grammars}
\label{sec:week4}

In Week 4, we delved into the concepts of parsing and context-free grammars, exploring how they are used to translate concrete syntax into abstract syntax. This is a crucial step in understanding how programming languages are processed and interpreted.

\subsection*{Key Concepts}

\begin{itemize}
    \item \textbf{Concrete syntax}: Represents a program as a string (e.g., "1 + 2 * 3")
    \item \textbf{Abstract syntax}: Represents a program as a tree structure
    \item \textbf{Parsing}: The process of transforming concrete syntax into abstract syntax
    \item \textbf{Context-free grammar}: A set of rules that define the structure of a language
\end{itemize}

\subsection*{Context-Free Grammar for Arithmetic Expressions}

We studied the following context-free grammar for arithmetic expressions:

\begin{verbatim}
Exp -> Exp '+' Exp1 
Exp1 -> Exp1 '*' Exp2              
Exp2 -> Integer            
Exp2 -> '(' Exp ')'  
Exp -> Exp1             
Exp1 -> Exp2
\end{verbatim}

\subsection*{Homework Solutions: Week 4}

For the homework, we were asked to parse various expressions using the given context-free grammar. The step-by-step derivations for each problem are figures 1-5 after the conclusion.

\begin{figure}[h]
\centering
\includegraphics[width=\textwidth, page=1]{DerivationTrees.pdf}
\caption{Derivation Tree for Expression 1}
\end{figure}

\begin{figure}[h]
\centering
\includegraphics[width=\textwidth, page=2]{DerivationTrees.pdf}
\caption{Derivation Tree for Expression 2}
\end{figure}

\begin{figure}[h]
\centering
\includegraphics[width=\textwidth, page=3]{DerivationTrees.pdf}
\caption{Derivation Tree for Expression 3}
\end{figure}

\begin{figure}[h]
\centering
\includegraphics[width=\textwidth, page=4]{DerivationTrees.pdf}
\caption{Derivation Tree for Expression 4}
\end{figure}

\begin{figure}[h]
\centering
\includegraphics[width=\textwidth, page=5]{DerivationTrees.pdf}
\caption{Derivation Tree for Expression 5}
\end{figure}

The parsing process demonstrates how the grammar rules are applied to derive the final expression, showing the hierarchical structure of the arithmetic operations.

\subsection*{Lessons Learned}

In Week 4, I gained several important insights:

\begin{enumerate}
    \item The importance of context-free grammars in defining the structure of programming languages
    \item How parsing bridges the gap between concrete syntax (what we write) and abstract syntax (how the computer interprets it)
    \item The hierarchical nature of expressions and how this is captured in abstract syntax trees
    \item The role of parsing in compiler design and language processing
\end{enumerate}

This week's content has deepened my understanding of how programming languages are structured and interpreted, providing a foundation for more advanced topics in language design and implementation.

\section{Week 5: Logic Game Tutorial and Lecture Content}
\label{sec:week5}

In Week 5, we focused on the content covered in the lectures and completed the Lean logic game's tutorial world. Below are the solutions for Levels 1 through 8, along with a proof for Level 8 written in mathematical logic.

\subsection*{Solutions: Lean Logic Game Tutorial}

\begin{enumerate}
    \item \textbf{Level 1:}
    \begin{lstlisting}[style=leanstyle]
example (P : Prop)(todo_list : P) : P := by
exact todo_list
    \end{lstlisting}

    \item \textbf{Level 2:}
    \begin{lstlisting}[style=leanstyle]
example (P S : Prop)(p: P)(s : S) : P ∧ S := by
exact ⟨p, s⟩
    \end{lstlisting}

    \item \textbf{Level 3:}
    \begin{lstlisting}[style=leanstyle]
example (A I O U : Prop)(a : A)(i : I)(o : O)(u : U) : (A ∧ I) ∧ O ∧ U := by
exact ⟨⟨a, i⟩, o, u⟩
    \end{lstlisting}

    \item \textbf{Level 4:}
    \begin{lstlisting}[style=leanstyle]
example (P S : Prop)(vm: P ∧ S) : P := by
exact vm.left
    \end{lstlisting}

    \item \textbf{Level 5:}
    \begin{lstlisting}[style=leanstyle]
example (P Q : Prop)(h: P ∧ Q) : Q := by
exact h.right
    \end{lstlisting}

    \item \textbf{Level 6:}
    \begin{lstlisting}[style=leanstyle]
example (A I O U : Prop)(h1 : A ∧ I)(h2 : O ∧ U) : A ∧ U := by
exact ⟨h1.left, h2.right⟩
    \end{lstlisting}

    \item \textbf{Level 7:}
    \begin{lstlisting}[style=leanstyle]
example (C L : Prop)(h: (L ∧ (((L ∧ C) ∧ L) ∧ L ∧ L ∧ L)) ∧ (L ∧ L) ∧ L) : C := by
exact h.left.right.left.left.right
    \end{lstlisting}

    \item \textbf{Level 8:}
    \begin{lstlisting}[style=leanstyle]
example (A C I O P S U : Prop)
(h: ((P ∧ S) ∧ A) ∧ ¬I ∧ (C ∧ ¬O) ∧ ¬U) : A ∧ C ∧ P ∧ S := by
exact ⟨h.left.right, h.right.right.left, h.left.left.left, h.left.left.right⟩
    \end{lstlisting}
\end{enumerate}

\subsection*{Level 8: Formal Proof in Mathematical Logic}

We want to show: If \(((P \land S) \land A) \land \neg I \land (C \land \neg O) \land \neg U\), then \(A \land C \land P \land S\).

\textbf{Proof:}

1. From \(((P \land S) \land A)\), we have:
   - \(P \land S\) (left side)
   - \(A\) (right side)

2. From \(P \land S\), we have:
   - \(P\)
   - \(S\)

3. From \(C \land \neg O\), we have \(C\).

4. Therefore, combining \(A\), \(C\), \(P\), and \(S\), we get \(A \land C \land P \land S\).

\subsection*{Discussion Question on Discord}

During this week's assignment, I asked the following question on Discord:

\begin{quote}
\small
"How scalable is Lean if applied to large-scale software verification projects?"
\end{quote}

\section{Week 6: Implication in Lean Logic}
\label{sec:week6}

In Week 6, we delved into the concept of implication using the Lean Logic game tutorial. This week focused on understanding and applying implication in logical propositions.

\subsection*{Key Concepts}

\begin{itemize}
    \item Implication in propositional logic
    \item Lambda functions in Lean
    \item Conjunction (\(\land\)) and its relationship with implication
    \item Function composition in logical proofs
\end{itemize}

\subsection*{Lean Logic Game Tutorial Solutions}

We completed levels 1-9 of the Lean Logic game tutorial, focusing on implication. Here are the solutions:

\begin{enumerate}
    \item \textbf{Level 1:}
    \begin{lstlisting}[style=leanstyle]
example (P C: Prop)(p: P)(bakery_service : P → C) : C := by 
exact bakery_service p
    \end{lstlisting}

    \item \textbf{Level 2:}
    \begin{lstlisting}[style=leanstyle]
example (C: Prop) : C → C := by
exact λ h : C, h
    \end{lstlisting}

    \item \textbf{Level 3:}
    \begin{lstlisting}[style=leanstyle]
example (I S: Prop) : I ∧ S → S ∧ I := by
exact λ h : I ∧ S, ⟨h.right, h.left⟩
    \end{lstlisting}

    \item \textbf{Level 4:}
    \begin{lstlisting}[style=leanstyle]
example (C A S: Prop) (h1 : C → A) (h2 : A → S) : C → S := by
exact λ c : C, h2 (h1 c)
    \end{lstlisting}

    \item \textbf{Level 5:}
    \begin{lstlisting}[style=leanstyle]
example (P Q R S T U: Prop) (p : P)
(h1 : P → Q) (h2 : Q → R) (h3 : Q → T)
(h4 : S → T) (h5 : T → U) : U := by
exact h5 (h3 (h1 p))
    \end{lstlisting}

    \item \textbf{Level 6:}
    \begin{lstlisting}[style=leanstyle]
example (C D S: Prop) (h : C ∧ D → S) : C → D → S := by
exact λ c d, h ⟨c, d⟩
    \end{lstlisting}

    \item \textbf{Level 7:}
    \begin{lstlisting}[style=leanstyle]
example (C D S: Prop) (h : C → D → S) : C ∧ D → S := by
exact λ h1 : C ∧ D, h h1.left h1.right
    \end{lstlisting}

    \item \textbf{Level 8:}
    \begin{lstlisting}[style=leanstyle]
example (C D S : Prop) (h : (S → C) ∧ (S → D)) : S → C ∧ D := by
exact λ s : S, ⟨h.left s, h.right s⟩
    \end{lstlisting}

    \item \textbf{Level 9:}
    \begin{lstlisting}[style=leanstyle]
example (R S : Prop) : R → (S → R) ∧ (¬S → R) := by
exact λ r : R, ⟨λ _ , r, λ _ , r⟩
    \end{lstlisting}
\end{enumerate}

\subsection*{Reflections and Insights}

We saw how lambda functions can be used to construct implications and how logical statements can be built and proven using basic principles of propositional logic.

The most intersting was the interaction between conjunction (\(\land\)) and implication (\(\rightarrow\)), as seen in levels 6 and 7. These exercises showed how we can convert between statements involving conjunctions and nested implications.

\section*{Conclusion}
In week one, we reviewed our discrete math knowledge and began coding proofs. Week two we saw the relationship between mathematical proofs and Lean code proofs, and then we began to solve problems recursively. In week three, I took a dive into quantum computing languages in my literature review while also working on coding a calculator using a recursive approach to solve parentheses. In week four, we explored parsing and context-free grammars, learning how to translate concrete syntax into abstract syntax, which is crucial for understanding how programming languages are processed. Week 5 used Lean as a way of proving logic puzzles. Week 6 built upon our logical foundations by focusing on implication in Lean Logic, proving logical statements.

\end{document}